%!TEX root = main.tex
\section*{Abstract}
\addcontentsline{toc}{section}{Abstract}
In English, 150〜300ワード、英語の図面1枚を含む.
\clearpage

\section*{和文概要}
\addcontentsline{toc}{section}{和文概要}
300〜400文字、日本語の図面1枚を含む.
\clearpage

\section{論文の構成}
\subsection{序論}
目的・テーマの意義を記述する.

例
\begin{itemize}
  \item{研究・制作の背景、動機、目的、意義などを述べる}
  \item{自分が対象とする分野において、 これまでどのような研究/制作が行われてきたか}
  \item{どのような問題・課題があり、その問題/課題になぜ取り組もうと思ったのか}
  \item{研究・制作の位置付け(既存研究・制作との関連性と違い)}
  \item{研究・制作の目的}
\end{itemize}

\subsection{本論}
適宜フォーマットを変更して記述する.

例
\begin{itemize}
  \item{研究・制作の手法、分析の内容仮説など}
  \item{研究・制作の結果と考察}
\end{itemize}

\subsection{結論}
例
\begin{itemize}
  \item{研究・制作の価値・まとめなど}
\end{itemize}

\subsection{文献}
「2.6文献」のスタイルに従って記述すること.

\subsection{謝辞}

\section{原稿の作成方法について}
ここでは,原稿を執筆する際に必要なことを解説します.

\subsection{英文概要}
1ページ以内.

\subsection{和文概要}
1ページ以内.

フォントサイズ 11、フォント MS明朝

\subsection{標準ページ数}
本文のレイアウト(1頁あたりの文字数)は,40字×25行=1,000字となります.
\begin{enumerate}
  \item{制作物が無い場合(20ページ以上、20,000文字程度)※表紙、目次、概要(英語版と日本語版)は除く}
  \item{制作物がある場合 小論文(5ページ以上、5000文字程度)※表紙、目次、概要(英語版と日本語版)は除く}
\end{enumerate}
  
\subsection{用字用語}
\begin{enumerate}
  \item{用字は原則として「常用漢字」を用い,仮名「新仮名づかい」とします.}
  \item{用語は原則として以下によるものとします.}
  \begin{enumerate}
    \item{「文部省学術用語集,電気工学編」及び本会編}
    \item{「電子情報通信用語辞典」}
    \item{「電子情報通信ハンドブック」}
  \end{enumerate}
  \item{量記号,単位記号の略号(SI)及びシンボルは,原則として本会編「電子情報通信ハンドブック」によるものとします.}
  \item{句読点は,句点「.」と読点「,」をそれぞれ全角で用います.}
\end{enumerate}

\subsection{図,写真,表}
\begin{enumerate}
  \item{図,写真,表は著者がオリジナルに作成したものを使用して下さい.}
  \item{すべての図,写真,表には,和英両方の題名(キャプション)を付けて下さい.}
  \item{図中の用語は原則として英文を用いて下さい.本文中で図中の英文用語に対応する和文用語を用いる場合には,必要に応じて当該和文用語の後に対応する英文用語を括弧に入れて示して下さい.}
\end{enumerate}

\begin{figure}[htbp]
\centering
\includegraphics[width=25mm]{gray.jpg}
\caption{画像の挿入例,Example of image insertion}
\end{figure}

\begin{table}[htbp]
\caption{表の挿入例,Example of table insertion}
\begin{center}
\begin{tabular}{|l|r|} \hline
a & 1 \\ \hline
b & 2 \\ \hline
 \end{tabular}
 \end{center}
 \end{table}
 
\subsection{文献}
文献は,以下のスタイルに従ってリストし,引用して下さい. 
\subsubsection{文献のリスト法}
\begin{enumerate}
  \item{付録Eの「学術雑誌略語表」(https://www.ieice.org/jpn/shiori/pdf/furoku\_e.pdf)に掲載されている雑誌名は,同表に従って略語で記す.}
  \item{著者が複数の場合には,全著者の氏名を記入する.なお,欧文の場合にはイニシャルと姓名を記入し, A.G. Wine のようにイニシャルと姓名の間にのみ半角スペースを挿入する.}
  \item{英文論文の標題中の単語については,文頭以外は小文字を使用する.}
  \item{欧文文献においては,常に半角ピリオド「.」と半角カンマ「,」を用いる.和文文献においては,読点には全角の「,」を用い,「vol.」,「no.」,「pp.」あるいは月名等の省略記号及び行末の句点には半角ピリオド「.」を用いる.なお,vol.J62-B,no.1,pp.20-27等の場合には,半角ピリオド「.」の後ろにはスペースは挿入しない.}
  \item{発行の年月を記載する場合には,月年の順で,月名には英語を,年には西暦を用いる.}
  \item{Webページは改版や消滅の可能性があるので,URLを参照することはできるだけ避ける.ただし,標準化団体などが文書公表の場をWebページにしている場合などはこの限りではない.}
  \item{和文表記の文献については,その英語表記を併記する.具体的には次の通りとする.}
  \begin{enumerate}
    \item{文献に英訳が存在する場合,次の2つのいずれかの方法を選択し,これに従って,記載を行なう.}
    \begin{enumerate}
      \item{和文表記の文献を記載し,改行の上,その英文表記の文献を併記する.}
      \item{和文表記の文献の代わりに英文表記の文献を記載する.}
    \end{enumerate}
  \item{文献に英訳が存在しない場合,和文表記の文献を記載し,改行の上,その和文の題目,雑誌名などをローマ字表記としたものを併記する.}
  \end{enumerate}
\end{enumerate}